\documentclass[12pt]{article}
\usepackage[a4paper, margin=1in]{geometry}
\usepackage{listings}
\usepackage{xcolor}
\usepackage{titlesec}
\usepackage{hyperref}

\titleformat{\section}{\normalfont\Large\bfseries}{Lab \thesection:}{1em}{}

\definecolor{codegray}{rgb}{0.5,0.5,0.5}
\definecolor{backcolor}{rgb}{0.95,0.95,0.92}
\definecolor{darkgreen}{rgb}{0.0, 0.5, 0.0}

\lstdefinestyle{mystyle}{
    backgroundcolor=\color{backcolor},   
    commentstyle=\color{darkgreen},
    keywordstyle=\color{blue},
    numberstyle=\tiny\color{codegray},
    stringstyle=\color{red},
    basicstyle=\ttfamily\footnotesize,
    breaklines=true,
    captionpos=b,
    numbers=left,
    numbersep=5pt,
    showstringspaces=false,
    tabsize=2
}
\lstset{style=mystyle}

\title{\textbf{Ubuntu Basics Lab Tutorial}}
\author{}
\date{}

\begin{document}

\maketitle

\section{Introduction to Ubuntu}
Ubuntu is a popular Linux distribution based on Debian. It is known for its user-friendliness, security, and wide range of support for hardware and software.

\textbf{Objective:} Get familiar with Ubuntu basics and its environment.

\section{Understanding the UI}

Ubuntu uses the \textbf{GNOME Desktop Environment} by default. GNOME provides a clean, efficient, and modern interface designed to be easy to use, even for those new to Linux.

\subsection*{GNOME Highlights:}
\begin{itemize}
    \item \textbf{Activities Overview}: A dynamic workspace management area that shows open windows, virtual desktops (workspaces), and a search bar.
    \item \textbf{Top Bar}: Contains system menus (date, network, volume), app shortcuts, and the system tray.
    \item \textbf{Dock}: A quick launcher bar on the left with favorite applications and currently running ones.
    \item \textbf{Search}: Simply press \texttt{Super (Windows key)} and type to search applications, files, and settings.
\end{itemize}

\subsection*{Steps to Explore the GNOME UI:}
\begin{enumerate}
    \item Log in to your Ubuntu desktop.
    \item Move the mouse to the top-left corner or press the \texttt{Super} key to open the Activities Overview.
    \item Click on the application launcher (``9-dot'' grid icon) to see all installed apps.
    \item Right-click any app in the Dock to pin/unpin it.
    \item Explore the system status by clicking on the top-right menu (for volume, network, power, etc.).
\end{enumerate}

\section{Major Default Applications in Ubuntu}

Ubuntu ships with a range of default applications to support daily tasks out of the box:

\begin{itemize}
    \item \textbf{Firefox} \textendash{} Web browser
    \item \textbf{LibreOffice Suite} \textendash{} Word Processor, Spreadsheet, Presentation (Writer, Calc, Impress)
    \item \textbf{Files (Nautilus)} \textendash{} File manager
    \item \textbf{Ubuntu Software Center} \textendash{} Graphical package installer and manager
    \item \textbf{Terminal} \textendash{} Command-line access
    \item \textbf{Settings} \textendash{} Configure system preferences
    \item \textbf{Rhythmbox} \textendash{} Music player
    \item \textbf{Videos (Totem)} \textendash{} Video playback application
    \item \textbf{Document Scanner} \textendash{} Scan documents with connected devices
    \item \textbf{Help} \textendash{} GNOME and Ubuntu help documentation
\end{itemize}

\section{Alternative UI Options}
Ubuntu supports various desktop environments such as:
\begin{itemize}
    \item KDE (Kubuntu)
    \item XFCE (Xubuntu)
    \item LXDE (Lubuntu)
\end{itemize}

\textbf{Command to install XFCE:}
\begin{lstlisting}[language=bash]
sudo apt install xubuntu-desktop
\end{lstlisting}

\section{Alternatives for Ubuntu}
Other Linux distributions include:
\begin{itemize}
    \item Debian
    \item Fedora
    \item Arch Linux
    \item Linux Mint
\end{itemize}

Each distribution has unique features, package managers, and user interfaces.

\section{Users and Permissions}
Linux is a multi-user system. Each file has associated permissions for:
\begin{itemize}
    \item \texttt{r} - Read
    \item \texttt{w} - Write
    \item \texttt{x} - Execute
\end{itemize}

\textbf{Check permissions:}
\begin{lstlisting}[language=bash]
ls -l filename
\end{lstlisting}

\textbf{Change permissions:}
\begin{lstlisting}[language=bash]
chmod u+x script.sh
\end{lstlisting}

\section{Super User and Privilege Escalation}

In Ubuntu and other Linux distributions, some operations require elevated privileges (administrative rights). There are two primary commands used for privilege escalation:

\subsection*{1. \texttt{sudo} (Superuser Do)}
\begin{itemize}
    \item Temporarily runs a single command as the superuser (or another user if specified).
    \item You need to be part of the \texttt{sudo} group.
    \item It uses your own password, not the root password.
\end{itemize}

\textbf{Example:}
\begin{lstlisting}[language=bash]
sudo apt update
sudo nano /etc/hosts
\end{lstlisting}

\subsection*{2. \texttt{su} (Switch User)}
\texttt{su} is used to switch to another user account or to become the root user.

\textbf{Usages:}
\begin{itemize}
    \item \texttt{su} \textendash{} Switches to root. Requires the root password (not available by default in Ubuntu).
    \item \texttt{su username} \textendash{} Switches to another specified user.
    \item \texttt{su -} \textendash{} Switches to root with the full login shell (loads environment variables like root's PATH).
    \item \texttt{su - username} \textendash{} Logs in as a specified user with a full login shell.
\end{itemize}

\textbf{Note:} In Ubuntu, the root account is locked by default. Users use \texttt{sudo} instead.

\textbf{Examples:}
\begin{lstlisting}[language=bash]
# Become root user (not typical in Ubuntu)
su

# Switch to another user
su john

# Switch to root with login shell (if root is unlocked)
su -

# Switch to user 'admin' with login shell
su - admin
\end{lstlisting}

\subsection*{3. Unlocking the Root Account (Optional)}
To set a password for the root user and enable direct login (not recommended for most users):

\begin{lstlisting}[language=bash]
sudo passwd root
# Enter new password when prompted
\end{lstlisting}

Once done, you can use:
\begin{lstlisting}[language=bash]
su -
# Then enter the root password
\end{lstlisting}

\subsection*{Best Practices:}
\begin{itemize}
    \item Use \texttt{sudo} for administrative tasks instead of \texttt{su} to follow Ubuntu's security model.
    \item Avoid logging in as root directly unless absolutely necessary.
    \item Always be cautious with commands run as superuser \textendash{} they can affect the entire system.
\end{itemize}

\section{Network, Bluetooth and Display Settings}

\subsection*{Network}
\begin{itemize}
    \item Open \texttt{Settings \textrightarrow{} Network}.
    \item Configure wired or wireless connections.
    \item Use the command line tool: \texttt{nmcli} for advanced network configuration.
\end{itemize}

\subsection*{Bluetooth}
\begin{itemize}
    \item Navigate to \texttt{Settings \textrightarrow{} Bluetooth}.
    \item Toggle Bluetooth ON, then pair with nearby devices.
\end{itemize}

\subsection*{Display}
\begin{itemize}
    \item Go to \texttt{Settings \textrightarrow{} Displays}.
    \item Adjust resolution, refresh rate, orientation, and scale.
    \item Set up extended or mirrored displays if multiple monitors are connected.
\end{itemize}

\section{General Settings}
Within the Settings application, explore and configure:
\begin{itemize}
    \item \textbf{Date \& Time} \textendash{} Configure manually or sync with network time.
    \item \textbf{Region \& Language} \textendash{} Change keyboard input and locale.
    \item \textbf{Power} \textendash{} Set sleep, screen blank, and battery thresholds.
    \item \textbf{Privacy} \textendash{} Manage screen lock, location services, and usage history.
\end{itemize}

\section{Workspaces}
Workspaces provide multiple virtual desktops for organizing your tasks.

\subsection*{How to Use Workspaces:}
\begin{itemize}
    \item Press \texttt{Super} key and view the workspace overview.
    \item Drag windows to other workspaces or create a new one by moving a window to an empty space.
    \item Navigate between workspaces using \texttt{Ctrl + Alt + Up/Down} or gestures on touchpads.
\end{itemize}

\section*{Conclusion}
This lab introduced Ubuntu's user interface, key system tools, default applications, user permissions, and system settings. These skills are essential for navigating and managing Linux-based operating systems effectively.

\end{document}